\documentclass{scrartcl}

\usepackage{fontspec}
%\usepackage{amsfonts}
\usepackage{ngerman}
\usepackage{verbatim}
\usepackage{url}
\usepackage{parskip}
\usepackage{fancyvrb}
\usepackage{xcolor}

%Definitionen für den Formatierer
\include{definitions}

\title{Lösung der Aufgabe~4}
\subtitle{des 31.~Bundeswettbewerbs Informatik}
\author{Tobias Bucher, Robert Clausecker}

% Syntaxdefinitionen
\setlength{\parskip}{0.5em plus0.5em minus0.5em}
\setlength{\parindent}{1em}

\newcommand{\src}[1]{\texttt{#1}}

\begin{document}
\maketitle

\section{Vorbetrachtung}
Gegeben ist eine Iteration des \textsc{Sierpinski}-Teppichs. Ziel ist es, die
nächste Iteration daraus zu erzeugen.

\section{Algorithmus}
Der Algorithmus zur Lösung dieser Aufgabe ist relativ simpel, man gehe alle
Elemente des SVGs durch und ersetze jedes weiße Quadrat durch neun kleinere
kongruente Quadrate, von denen das mittlere schwarz ist.

\section{Ausgabe}
Siehe Abbildungen.

\newcommand{\SierpinskyIteration}[1]{%
\begin{figure} %
	\center\includegraphics[width=0.9\textwidth]{figur0_it_#1} %
	\caption{Iteration #1 des \textsc{Sierpinski}-Teppichs} %
\end{figure}
}

\SierpinskyIteration{1}
\SierpinskyIteration{2}
\SierpinskyIteration{3}
\SierpinskyIteration{4}
\SierpinskyIteration{5}
\SierpinskyIteration{6}

\newpage\appendix
\section{Quelltext}
Es folgt der Quelltext für das Iterieren des \textsc{Sierpinsky}-Teppichs. Das
Programm ist in Python 3 unter Linux geschrieben worden, sollte aber auch auf
anderen Betriebssystemen ohne Änderungen laufen.

Das Programm nimmt optional eine Eingabedatei und eine Ausgabedatei als
Parameter.

Beispiel für einen möglichen Aufruf: \src{./iterate.py figur0.svg}

\newcommand{\InputSource}[1]{\subsection*{#1.py}\input{#1}}

\InputSource{iterate}

\end{document}
