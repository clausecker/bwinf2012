\documentclass{scrartcl}

\usepackage{libertineotf}
\usepackage[euler-digits,small]{eulervm}

\usepackage{fontspec}
%\usepackage{amsfonts}
\usepackage{ngerman}
\usepackage{verbatim}
\usepackage{url}
\usepackage{parskip}
\usepackage{fancyvrb}
\usepackage{xcolor}

%Definitionen für den Formatierer
\include{definitions}

\title{Lösung der Aufgabe~5}
\subtitle{des 31.~Bundeswettbewerbs Informatik}
\author{Tobias Bucher, Robert Clausecker}

% Syntaxdefinitionen
\setlength{\parskip}{0.5em plus0.5em minus0.5em}
\setlength{\parindent}{1em}

\begin{document}
\maketitle
\section{Vorüberlegungen}
Die Kaffekanne fässt zehn Tassen, sie muss also nachdem zehn Mitarbeiter Kaffe
getrunken haben aufgefüllt werden. Daher ist zu erwarten, dass ein Mitarbeiter
im Schnitt alle zehn Male, die er Kaffe trinkt die Kanne auffüllen muss. Ich
erwarte daher, dass ein Schwellwert $\alpha<\frac1{10}$ dazu führt, dass kaum
einer bzw. gar keiner der Mitarbeiter Kaffee trinken wird, weil dieser Wert im
Mittel nicht erfüllt werden kann. Für Schwellwerte $\alpha$, die ein wenig
größer sind, erwarte ich, dass Mitarbeiter meistens glücklich sind, aufgrund
statistischer Streungen wird es aber immer Situationen geben, wo Mitarbeiter
nicht glücklich sind. Je höher $\alpha$ ist, desto seltener wird diese Situation
sein.

\section{Simulation}
Die Simulation wird von einem Programm durchgeführt. Es als Variable Parameter
sind die Anzahl der Tage, die simuliert werden und der Schwellwert $\alpha$ zu
betrachten. Ersterer Parameter hat kaum Einfluss auf die Simulationsergebnisse,
es ist aber darauf zu achten, diesen hinreichend groß zu wählen, um den Einfluss
der ersten zehn Tage auf die Statistik zu mindern. Ich habe hier $2^{20}$
gewählt; ein Durchlauf für diesen Wert läuft auf meinem System in knapp einer
Sekunde und ist ausreichend lang.

Zu betrachten ist auch die Auswahl von Kandidaten für $\alpha$. Offensichtlich
kommt nur der Bereich $0\le\alpha\le1$ in Betracht. Ein Wert von $\alpha>1$ ist
nicht sinnvoll, weil es unmöglich ist häufiger Kaffee zu kochen als diesen zu
trinken und damit die Bedingung fürs unglücklich sein niemals erfüllt werden
kann.

Das nächste Detail ist die Granularität. Welche Werte für die Anzahl der
getrunkenen Tassen $n$ und gebrühter Kannen $m$ kommen in Frage? Jeder der
Angestellten versucht am Tag drei Tassen zu trinken, in zehn Tagen also 30. Dies
ist auch die obere Schranke von $m$. Es lohnt sich für $\alpha$ nur Brüche mit
Nenner und Zähler aus diesem Bereich zu testen. Da es die Vorbedinung $n\ge10$
gibt, bleiben für $n$ und $m$ die ganzzahligen Bereiche $n\in\{10\dots30\}$ und
$m\in\{0\dots30\}$ übrig. Dies sind insgesamt 600 Brüche. Ich habe in der
Simulation die 626 Werte $\alpha\in\{0,\frac1{625},\frac2{625},\dots,1\}$
verwendet. Nach dem Schubfachprinzip ist jeder mögliche Bruch in dieser
Testmenge enthalten.

Simulationen für alle $\alpha$ aus oben beschriebener Testmenge durchzuführen
dauert auf meinem System anderthalb Minuten. Ich erwarte keine signifikante
Verbesserung der Ergebnisse durch mehr Fälle für $\alpha$ oder mehr Tage.

\section{Ergebnisse}
\begin{figure}
\begin{center}
\includegraphics{stats.pdf}
\end{center}
\caption{Ergebnis der Simulation}
\label{simulation}
\end{figure}

Es zeigt sich, dass für $\alpha>\frac12$ die Anzahl der getrunkenen Tassen
Kaffee immer maximal ist. Aus diesem Grund habe ich die Beispielsimulation
in Abbildung \ref{simulation} nur im Bereich $\alpha\in[0,\frac12]$ dargestellt.

Die Ergebnisse der Simulation (vgl. Abbildung \ref{simulation}) erfüllen die
vorher gestellten Erwartungen. Deutlich erkennbar ist der sofortige Sprung der
getrunkenen Tassen bei $\alpha=\frac1{10}$. Interessant ist eine kleine
statistische Anomalie: im Bereich $\alpha\in[0.08,0.10]$ springt die Anzahl der
getrunkenen Tassen mit steigendem $\alpha$ hoch und runter. Ich vermute, dass
dies an bestimmten chaotisch ablaufenden Ketten von Wahrscheinlichkeiten liegt.

\appendix
\section{Quelltext}
Es folgt der Quelltext des geschriebenen Simulationsprogrammes. Dieses ist in
C99 für das Betriebssystem Linux geschrieben. Das Programm sollte auch auf
anderen Unix-Derivaten ohne Änderungen laufen. Um es auf gänzlich anderen
Systemen auszuführen ist es unter Umständen notwendig, die Initialisierung des
Zufallsgenerators zu verändern. Das Programm liest zu diesem Zweck die Datei
\texttt{/dev/urandom} aus, welche auf anderen Betriebssystemen nicht verfügbar
sein könnte.

\include{stats}
\end{document}
