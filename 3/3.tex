\documentclass{scrartcl}

\usepackage{fontspec}
%\usepackage{amsfonts}
\usepackage{ngerman}
\usepackage{verbatim}
\usepackage{url}
\usepackage{parskip}
\usepackage{fancyvrb}
\usepackage{xcolor}

%Definitionen für den Formatierer
\include{definitions}

\title{Lösung der Aufgabe~3}
\subtitle{des 31.~Bundeswettbewerbs Informatik}
\author{Tobias Bucher, Robert Clausecker}

% Syntaxdefinitionen
\setlength{\parskip}{0.5em plus0.5em minus0.5em}
\setlength{\parindent}{1em}

\newcommand{\src}[1]{\texttt{#1}}

\begin{document}
\maketitle

\section{Vorbetrachtungen}
Das gegebene Labyrinth besteht aus einer quadratisch gerasterten endlichen
Fläche, in der jedes Quadrat des Gitters eindeutig entweder betretbar oder
nicht betretbar ist. Der Roboter "`Turn90"' hat nur eingeschränkte Bewegungs-
und Wahrnehmungsmöglichkeiten, so kann er sich nur um 90 Grad drehen und
vorwärts bewegen. Weiterhin hat er Kontaktsensoren, die bestimmen können, ob
der Roboter sich an einer Wand befindet.

Vereinfacht werde angenommen, dass der Roboter sich immer auf genau einem Feld
in dem Raster befindet. Er habe die Möglichkeit, sich um 90 Grad zu drehen, ein
Feld vorwärts zu gehen, das Feld vor ihm und das Feld links von ihm auf
Kollision prüfen.

\section{Grundidee}
Um den Ausgang des Raums zu finden, bietet sich ein Algorithmus zum Lösen von
Irrgarten an, da der gerasterte Raum auch nur ein Irrgarten mit unüblich vielen
Gängen ist. Da der Roboter irgendwo im Raum startet und der Ausgang an der
Außenmauer ist, bietet sich hierfür der \textsc{Pledge}-Algorithmus an.

\section{Algorithmus}
Der \textsc{Pledge}-Algorithmus funktioniert für ein rechteckiges Raster wie
folgt, wobei nach jeder Bewegung geprüft werden soll, ob der Roboter sich
bereits am Ziel befindet:

\begin{enumerate}
\item Laufe geradeaus bis zu einem Hindernis
\item In diesem Schritt wird ein Zähler eingeführt, der die aktuelle Umdrehung
mitzählt. Verfahre wie folgt: Laufe nach rechts an der linken Wand entlang, bis
der Zähler wieder auf $0$ ist, wobei Folgendes zu beachten ist:
	\begin{itemize}
	\item Der Zähler wird um eins hochgezählt wird, wenn der Roboter sich aufgrund des Wandfolgens nach rechts dreht.
	\item Der Zähler wird um eins heruntergezählt, wenn er sich entsprechend nach links dreht.
	\end{itemize}
Wenn der Zähler wieder auf $0$ ist, gehe zu Schritt 1.
\end{enumerate}

\section{Beispielausgaben}
Siehe Abbildungen.

\newcommand{\RoomFigure}[1]{%
\begin{figure} %
	\center\includegraphics[width=0.9\textwidth]{#1} %
	\caption{\src{#1.txt}} %
\end{figure} %
}

\begin{figure}
	\center\includegraphics[width=0.9\textwidth]{raum0_beispiel}
	\caption{\src{raum0\_beispiel.txt}}
\end{figure}

\RoomFigure{raum1}
\RoomFigure{raum2}
\RoomFigure{raum3}
\RoomFigure{raum4}
\RoomFigure{raum5}

\newpage\appendix
\section{Quelltext}
Es folgt der Quelltext für das Lösen des Labyrinths, das Lesen des
spezifizierten Dateiformats und die Ausgabe. Das Programm ist in Python 3 unter
Linux geschrieben worden, sollte aber auch auf anderen Betriebssystemen ohne
Änderungen laufen.

Für die Ausgabe wird die Python-Bibliothek \src{tkinter} benutzt.

Das Programm nimmt die Eingabedatei als Parameter und optional noch einen
Skalierungsfaktor der Ausgabe, die angibt, wieviele Pixel ein Rasterfeld hoch
und breit ist.

Beispiel für einen möglichen Aufruf: \src{./turn90.py raum0\_beispiel.txt %
--scale 5}

\newcommand{\InputSource}[1]{\subsection*{#1.py}\input{#1}}

\InputSource{turn90}
\subsection*{grid\_canvas.py}\input{grid_canvas}
\InputSource{path}
\InputSource{room}
\InputSource{natives}

\end{document}
